    {\actuality} Из множества источников исторической информации, наиболее полные сведения о культурном наследии мировой
    истории содержатся в письменных источниках.
    Во многом благодаря письменным документам можно восстановить наиболее точную картину прошлого.
    Однако, исследования показали, что сохранение таких источников в надлежащем состоянии представляет собой значительную проблему~\cite{болдовская2023историческая}.
    Естественное старение, ненадлежащее хранение и небрежное обращение с письменными документами могут привести к их безвозвратной утрате, что наносит огромный ущерб культурному наследию.
    Более того, отсутствие единого стандарта при их создании приводит к значительному разнообразию видов и типов документов, что усложняет работу с ними в ходе исследований.

    Информационно-аналитический онлайн-ресурс «Православный ландшафт таежной Сибири» разрабатывается с целью решения,
    по крайней мере, нескольких этих проблем.
    Все исторические данные, загруженные в систему, будут оцифрованы и приведены к единому стандарту, что ускорит исторические
    исследования и упростит работу с данными.
    Пользователь сможет просматривать данные в удобном формате, будь то текстовый или графический вариант представления.
    Оцифровка также позволит сократить использование оригинальных письменных экземпляров, поскольку станет возможной
    работа с множеством источников в одном месте.
    Это способствует более длительному сохранению физических оригиналов в хорошем состоянии.
    Кроме того, оцифрованные копии исторических данных будут храниться в нескольких местах, что обеспечит их доступность и предотвратит безвозвратную утрату ценной информации.

        {\aim} данной работы является проектирование и разработка оптимальной инфраструктуры для работы
    микросервисов в проекте «Православный ландшафт таежной Сибири».

    Для~достижения поставленной цели необходимо было решить следующие {\tasks}:
    \begin{enumerate}[beginpenalty=10000] %https://tex.stackexchange.com/a/476052/104425
        \item Исследовать актуальные методологии и подходы к проектированию и внедрению микросервисов.
        \item Разработать архитектуру системы, включающую все необходимые компоненты и сервисы.
        \item Обеспечить интеграцию разработанной инфраструктуры с существующими системами и базами данных.
    \end{enumerate}
